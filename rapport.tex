\documentclass{article}
\usepackage[utf8]{inputenc}
\usepackage{listings}
\usepackage{color}

\title{DM Complexité et Calculabilité}
\author{Guillaume NEDELEC et Alfred Aboubacar SYLLA }
\date{Pour le 5 Novembre 2018}

%New colors defined below
\definecolor{codegreen}{rgb}{0,0.6,0}
\definecolor{codegray}{rgb}{0.5,0.5,0.5}
\definecolor{codepurple}{rgb}{0.58,0,0.82}
\definecolor{backcolour}{rgb}{0.95,0.95,0.92}

%Code listing style named "mystyle"
\lstdefinestyle{codeStyle}{
  backgroundcolor=\color{backcolour},   commentstyle=\color{codegreen},
  keywordstyle=\color{magenta},
  numberstyle=\tiny\color{codegray},
  stringstyle=\color{codepurple},
  basicstyle=\footnotesize,
  breakatwhitespace=false,         
  breaklines=true,                 
  captionpos=b,                    
  keepspaces=true,                 
  numbers=left,                    
  numbersep=5pt,                  
  showspaces=false,                
  showstringspaces=false,
  showtabs=false,                  
  tabsize=2
}

\lstset{style=codeStyle}

\begin{document}

\maketitle

\section{Définition du problème}

Soit \textit{$G = (V, E)$} un graphe non-orienté. Un triangle \textit{$T$} dans \textit{$G$} est une clique de taille 3,  autrement dit un ensemble \textit{$T = \{ u, v, w \}$}, tel que \textit{$u, v, w \in V$} sont distincts et \textit{$\{(u, v),(v, w),(w, u)\} \subseteq E$}.

Une partition de \textit{$G$} en triangles est une partition \textit{$V = T_{1} \: \cup \: T_{2} \: \cup \: . . . \: \cup \: T_{k}$} des sommets de \textit{$G$} telle que chaque \textit{$T_{i}$} est un triangle.

\smallbreak

\textbf{Problème} \textbf{\textit{Triangles}}

\textbf{Entrée} : Un graphe non-orienté G.

\textbf{Sortie} : Est-ce que G possède une partition en triangles ?

\section{Degré 3 }

Vous allez montrer que le problème $Triangles$ peut être résolu en temps $O(|V| + |E|)$ (linéaire) si le degré de chaque sommet est au plus 3. Le degré d’un sommet est le nombre de voisins.
\bigbreak

\noindent1. Que peut-on dire si $G$ contient un sommet de degré 1 ?
\smallbreak
\noindent \textbf{REPONSE :} Si $G$ contient un sommet de degré 1, alors il ne forme pas de triangle avec d'autres sommets. On peut en déduire que $G$ ne possède pas de partition en triangles.
\bigbreak

\noindent2. Supposez que $v$ est un sommet de degré 2. Montrez qu’on peut soit répondre “non” tout de suite, ou enlever des sommets de $G$, en obtenant $G_{0}$ tel que le $G$ est instance positive de $Triangles$ ssi $G_{0}$ l’est.
\smallbreak
\noindent \textbf{REPONSE :} Si le sommet $v$ est de degré 2, on peut repondre "non" tout de suite si et seulement les 2 sommets adjacents à $v$ ne sont pas reliés par une arête entre eux. Dans ce cas si on enleve des sommets de $G$ (différent de $v$ et de ses voisins), $G_{0}$ ne pourra pas être une instance positive de $Triangles$ étant donné que $v$ ne forme pas de triangle avec ses 2 voisins.
\bigbreak

\noindent3. On suppose maintenant que tous les sommets ont degré 3.
Les 4 cas de la figure ci-dessus indiquent quel est le voisinage possible du sommet $v$. Raisonnez comme au point précédent
\smallbreak
\noindent \textbf{REPONSE :} \smallbreak
\noindent\textbf{Graphe 1} : Les sommets adjacents de $v$ ne forment pas un triangle avec $v$. Donc même si ils forment des triangles avec d'autres voisins ou qu'on les retire, $v$ ne fera partie d'aucun triangle et donc $G_{1}$ sera une instance négative de $Triangles$.
\smallbreak
\noindent \textbf{Graphe 2} : Le sommets $v$ a deux sommets adjacents reliés par une arête. Ils forment donc un triangle. Pour vérifier que le reste du graphe est partitionné en triangles, on fait la même vérification sur le $3^{ieme}$ sommet adjacent à $v$ avec ses 2 autres voisins. Si il forme un triangle, alors le graphe $G_{2}$ est une instance positive sinon non.
\smallbreak
\noindent \textbf{Graphe 3}: Le graphe $G_{3}$ n'est pas une instance positive car les sommets adjacents de $v$ forment 2 triangles où $v$ appartient aux deux triangles. Les sommets supposés de degré 3 au maximum ne peuvent donc pas former d'autres triangles afin de n'inclure $v$ que dans 1. On en déduit donc que $G_{3}$ est une instance négative de $Triangles$ 
\smallbreak
\noindent \textbf{Graphe 4}: En suivant le même raisonnement qu'avec le graphe $G_{3}$, $G_{4}$ est une instance négative de $Triangles$ car tous les voisins de $v$ forme des triangles avec $v$, ne formant pas des triangles disjoints. De plus tous les voisins de $v$ sont de degré 3 et ne peuvent donc pas former d'autres triangles.
\bigbreak

\noindent4. Proposez un algorithme linéaire pour résoudre $Triangles$ sur les graphes de degré maximal au plus 3.
\smallbreak
\noindent \textbf{REPONSE :}
\smallbreak
\begin{lstlisting}[caption=Algortihme Triangles]
G = (V,E);
triangle = vrai;
i = 0;
Tant que (triangle == vrai ET i < |V|) {
    v = V[i];
    if (v.degre == 1 OU v.degre == 0) {
        triangle = faux;
    }
    else if (v.degre = 2) {
        Si il n'existe pas d'aretes entre les 2 voisins de v {
          triangle = faux;
        }
    }
    else if (v.degre = 3){
        Si il existe aucune OU plus d'une arete entre les 3 voisins de v {
            triangle = faux;
        }
    }
    i++;
}
retourner triangle;
\end{lstlisting}
\bigbreak
La compléxité de cet algorithme est de $O(|V| + |E|)$, cette complexité est linéaire.
\bigbreak

\section{Réduction de $Triangles$ vers SAT}

Afin d'effectuer une réduction du problème $Triangles$ vers $SAT$, nous avons utiliser des variables booléennes de la forme $x_{u,v}$.
Nous allons donner une orientation à chaque triangle, ce qui permet de définir un “successeur” pour chaque noeud. 

Par exemple, si \{1, 3, 6\} forme un triangle, alors une orientation possible est 3, 1, 6, le "successeur" de 1 étant 6, le “successeur” de 6 étant 3, et le “successeur” de 3 étant 1. 
La variable $x_{u,v}$ est vraie si v est le “successeur” de u (dans le sens décrit précédemment).

Pour effectuer la réduction nous avons donc découper le problèmes en plusieurs contraintes dont voici les formules :

\bigbreak

1. Un sommet est le successeur de son successeur 

$$ \bigwedge\limits_{u,v,w \in V; u \ne v \ne w} (\neg x_{u,v} \vee \neg x_{v,w} \vee x_{w,u}) $$
\textbf{Complexité : $O(|V|^3 + |E|^3)$}
\bigbreak

2. Pour chaque sommet $u \in V$, il n'existe qu'un sommet $v \in V$ où $v$ est successeur de $u$

$$ \bigwedge\limits_{u,v,w \in V; u \ne v \ne w} (\neg x_{u,v} \vee \neg x_{u,w}) $$
\textbf{Complexité : $O(|V|^3 + |E|^3)$}
\bigbreak

3. Pour chaque sommet $u \in V$, il n'existe qu'un sommet $v \in V$ où $u$ est successeur de $v$

$$ \bigwedge\limits_{u,v,w \in V; u \ne v \ne w} (\neg x_{u,v} \vee \neg x_{w,v}) $$
\textbf{Complexité : $O(|V|^3 + |E|^3)$}
\bigbreak

4. Avec $n$ le nombre de sommet de $G$, Si $n \bmod 3 \ne 0$ alors $G$ ne possède pas une partition en triangles 
\smallbreak
\noindent\textbf{Complexité : Constante}


\bigbreak
\noindent La complexité totale de la formule est donc :  \textbf{$O(3|V|^3 + 3|E|^3)$} soit \textbf{$O(|V|^3 + |E|^3)$}

\smallbreak
\noindent\textbf{CONCLUSION :} Cette réduction est donc une réduction polynomiale.
\end{document}
